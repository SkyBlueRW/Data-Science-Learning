

\documentclass{article}

\input{structure.tex} % Include the file specifying the document structure and custom commands

%----------------------------------------------------------------------------------------
%	ASSIGNMENT INFORMATION
%----------------------------------------------------------------------------------------

\title{High Performance Python } % Title of the assignment

\begin{document}
\maketitle
\tableofcontents


\section{Performant Python}
	
	\subsection{Computer System}
		\begin{enumerate}
			\item {\bf Computational Uinit} 
				\begin{itemize}
					\item Key properties are {\bf IPC} (instruction per cycle) and {\bf clock speed} (number of cycles per second).
					\item {\bf GIL (Global Interpreter Lock)} makes sure that a python process can run only one instruction at a time regardless of number of cores it is currently using. Can be avoided by multiprocessing, numpy, Cython or distributed models of computing.
				\end{itemize}
			\item {\bf Memory Unit} 
				\\ In terms of read/write speeds and latency, Spinning hard drive < Solid-state hard drive < RAM < L1/L2 cache
			\item {\bf Communication Layers}
		\end{enumerate}
	\subsection{Python Performance Dragger}
		\begin{enumerate}
			\item {\bf Not compiled} hence missing compiler tricks from source code to machine code
			\item {\bf Dynamic data type} more overhead before computation
			\item {\bf Garbage Collected Laguange \-> memory fragmentation}: Python objects are not laid out in the most optimal way in memory.
			\item {\bf GIL}: Only one instruction at a time
		\end{enumerate}


\section{Python Data Type}
	
	\subsection{List and \& Tuple}
	
	\subsection{Dictionary \& Set}

	\subsection{Iterator \& Generator}

	\subsection{Matrix \& Vector computation}

\end{document}